\chapter{第三章}

以下内容检查一行是否为 35 个汉字
\newline
一二三四五六七八九十一二三四五六七八九十一二三四五六七八九十一二三四五六七八九十一二三四五六七八九十一二三四五六七八九十一二三四五六七八九十

\section{引理 autorefname}

\begin{definition}[Lagrangian]\label{def.lag}
    Lagrangian 函数(定义不采用斜体)
\end{definition}

\begin{theorem}[ADMM]\label{th.admm}
    非精确 ADMM 的收敛性(定理环境采用斜体)
\end{theorem}

\begin{lemma}[PADMM]\label{th.padmm}
    非精确 PADMM 的收敛性
\end{lemma}

使用 \verb|\autoref{<label>}| 引用定义、定理、引理

\vspace{10bp}

\begin{minipage}{.3\textwidth}
    \verb|\autoref{def.lag}|\newline
    \verb|\autoref{th.admm}|\newline
    \verb|\autoref{th.padmm}|
\end{minipage}
~~~~
\begin{minipage}{.3\textwidth}
    \autoref{def.lag}\newline
    \autoref{th.admm}\newline
    \autoref{th.padmm}
\end{minipage}

\section{Math}

关于 \LaTeX 中的数学指令参考 \href{https://mirrors.sjtug.sjtu.edu.cn/ctan/info/short-math-guide/}{一份简短的 \LaTeX 数学指南}。
常用数学字体
\begin{itemize}
    \item 黑板粗体(双线体) $\setl{ABCDEFGHIJKLMNOPQRSTUVWXYZ}$
    \item 花体 $\setfaml{ABCDEFGHIJKLMNOPQRSTUVWXYZ}$
    \item 手写体 $\mapl{ABCDEFGHIJKLMNOPQRSTUVWXYZ}$
    \item 哥特体 $\mathfrak{ABCDEFGHIJKLMNOPQRSTUVWXYZ}$
\end{itemize}
% 使用 \verb|Package:eucal| 将替换手写体 "Computer Modern calligraphic" 为 "Euler script"


\section{参考文献 biblatex}

% 表格中的部分命令不能在参数 manualbib 下输出 (\documentclass[manualbib]{hnuthesis})
\begin{table}[ht]
    \caption{常用 cite 指令}
    \centering
    \begin{tabular}{l c l}
        \hline
        \verb|\parencite{golubMatrixComputations2013}| & $\Rightarrow$ & 引用\parencite{golubMatrixComputations2013} \\
        \verb|\parencite[31]{golubMatrixComputations2013}| & $\Rightarrow$ & 引用\parencite[31]{golubMatrixComputations2013} \\
        \verb|\cite{golubMatrixComputations2013}| & $\Rightarrow$ & 引用\cite{golubMatrixComputations2013} \\
        \verb|\cite[31]{golubMatrixComputations2013}| & $\Rightarrow$ & 引用\cite[31]{golubMatrixComputations2013} \\
        \verb|\upcite{golubMatrixComputations2013}| & $\Rightarrow$ & 引用\upcite{golubMatrixComputations2013} \\
        \verb|\textcite{golubMatrixComputations2013}| & $\Rightarrow$ & \textcite{golubMatrixComputations2013} \\
        \verb|\authornumcite{golubMatrixComputations2013}| & $\Rightarrow$ & \authornumcite{golubMatrixComputations2013} \\
        \verb|\citeauthor{golubMatrixComputations2013}| & $\Rightarrow$ & \citeauthor{golubMatrixComputations2013} \\
        \verb|\citeauthor{golubMatrixComputations2013}| & $\Rightarrow$ & \citeauthor{golubMatrixComputations2013} \\
        \hline
    \end{tabular}
\end{table}

\subsection{参考文献输出样式}

特别说明,基于 biblatex,本模板只能输出基本符合要求的参考文献列表,不能保证输出的参考文献完全符合 HNU 的要求。
如果发生无法解决的意外情况,参考\autoref{subsec.bibitem}。


\subsection*{重点关照的文献输出类型}
\begin{itemize}
    \item 硕博学位论文 \parencite{libaiThesis}
    \item arxiv 论文 \parencite{ishidaQuantitativeConvergenceDiscretization2023}
\end{itemize}

\subsection*{关于硕博学位论文}

按照 HNU 的要求,硕博学位论文的参考文献输出格式为 \fbox{[学校及学位论文级别].保存地点:保存单位}。
例如 \fbox{[湖南大学:博士学位论文]. 长沙:湖南大学}。
对于这样的输出要求,由于时间精力有限,暂时没法做到自动化输出参考文献。
一种简单的方式为手动修改 bib 文件中 \verb|@phdthesis| 条目的 type 和 school 域
\begin{itemize}
    \item \verb|type = {[湖南大学博士学位论文]}|
    \item \verb|school = {长沙:湖南大学}|
\end{itemize}

\subsection*{关于 arxiv 等在线论文}

要求的格式为 "<作者姓名>. <文献标题>. <线上网址>, <年份/日期>"。
一下给出一共参考示例 (基于 \parencite{ishidaQuantitativeConvergenceDiscretization2023} )

\begin{itemize}
    \item Ishida S, Lavenant H.
    Quantitative convergence of a discretization of dynamic optimal transport using the dual formulation.
    \href{https://arxiv.org/abs/2312.12213}{\nolinkurl {arxiv.org/abs/2312.12213}}, 2023.
\end{itemize}

\subsection{手动调整参考文献样式}\label{subsec.bibitem}
如果发生意外情况,参考文献的输出结果不符合 HNU 模板要求。
建议完成毕业论文后,再使用 biblatex 输出 bibitem,并手动微调 bibitem 的信息进行更细致的调整。
然而,这种微调方法的代价是,只能使用上标引用 \verb|\cite| 和非上标引用 \verb|\parencite| 两种文献引用类型,不能引用文献的作者、年份等等信息(因为我们不再使用 bib 数据库,而 bibitem 只有纯文本的输出信息,不能区分文献的作者、年份)。
首先输出 bibtiem
\begin{enumerate}
    \item 在文档结尾处使用命令 \verb|\printbibitembibliography| 将 biblatex 输出结果所对应的 bibitem 列表 \verb|`thebibliography`| 输出到 PDF 的对应位置
    % \begin{itemize}
    %     \item \verb|\printbibitembibliography| 来自于 Package: biblatex2bibitem,并且它已经在 \verb|hnuthesis.cls| 中引入。
    % \end{itemize}
    \item 将 bibitem 列表复制到单独的 tex 文件中(例如 \verb|bibitem.tex|),接着对其中的参考文献输出样式进行细微的调整。其中 bibitem 列表 \verb|`thebibliography`| 的形式为
    \begin{verbatim}
        \begin{thebibliography}{99}
            \bibitem{...}...
            \bibitem{...}...
            ... 
        \end{thebibliography}
    \end{verbatim}
\end{enumerate}
接着,根据 bibitem 引用参考文献
\begin{enumerate}
    \item \verb|\documentclass[..., manualbib]{hnuthesis}|,使用 manualbib 选项调用 hnu 模板类
    \item 通过(\verb|.cls| 中定义的)上标 \verb|\cite| 和非上标 \verb|\parencite| 在正文中引用
    \item 在 \verb|main.tex| 中使用 \verb|\begin{thebibliography}{99}
    {}
    \bibitem{he2016deep}
    He K, Zhang X, Ren S, et al. Deep residual learning for image recognition. in:
    Proceedings of the IEEE conference on computer vision and pattern recognition. 2016:
    770-778.
    {}
    \bibitem{libaiThesis}
    李柏. {复杂约束下自动驾驶车辆运动规划的计算最优控制方法研究}. [浙江大学博
    士学位论文]. 浙江:浙江大学, 2018.
    {}
    \bibitem{chenjinbiao1980jixian}
    陈晋镳, 张惠民, 朱士兴, 等. 蓟县震旦亚界研究. 见: 中国地质科学院天津地质矿产
    研究所. 中国震旦亚界. 天津: 天津科学技术出版社, 1980: 56-114.
    {}
    \bibitem{golubMatrixComputations2013}
    Golub G H, Van Loan C F. Matrix Computations. Fourth edition. Baltimore: The Johns
    Hopkins University Press, 2013. 756 pp.
    {}
    \bibitem{ishidaQuantitativeConvergenceDiscretization2023}
    Ishida S, Lavenant H. Quantitative convergence of a discretization of dynamic optimal transport using the dual formulation.
    \href{https://arxiv.org/abs/2312.12213}{\nolinkurl {arxiv.org/abs/2312.12213}}, 2023.
\end{thebibliography}
    | 输出参考文献
\end{enumerate}
